% Options for packages loaded elsewhere
\PassOptionsToPackage{unicode}{hyperref}
\PassOptionsToPackage{hyphens}{url}
%
\documentclass[
]{article}
\usepackage{amsmath,amssymb}
\usepackage{lmodern}
\usepackage{iftex}
\ifPDFTeX
  \usepackage[T1]{fontenc}
  \usepackage[utf8]{inputenc}
  \usepackage{textcomp} % provide euro and other symbols
\else % if luatex or xetex
  \usepackage{unicode-math}
  \defaultfontfeatures{Scale=MatchLowercase}
  \defaultfontfeatures[\rmfamily]{Ligatures=TeX,Scale=1}
\fi
% Use upquote if available, for straight quotes in verbatim environments
\IfFileExists{upquote.sty}{\usepackage{upquote}}{}
\IfFileExists{microtype.sty}{% use microtype if available
  \usepackage[]{microtype}
  \UseMicrotypeSet[protrusion]{basicmath} % disable protrusion for tt fonts
}{}
\makeatletter
\@ifundefined{KOMAClassName}{% if non-KOMA class
  \IfFileExists{parskip.sty}{%
    \usepackage{parskip}
  }{% else
    \setlength{\parindent}{0pt}
    \setlength{\parskip}{6pt plus 2pt minus 1pt}}
}{% if KOMA class
  \KOMAoptions{parskip=half}}
\makeatother
\usepackage{xcolor}
\usepackage[margin=1in]{geometry}
\usepackage{graphicx}
\makeatletter
\def\maxwidth{\ifdim\Gin@nat@width>\linewidth\linewidth\else\Gin@nat@width\fi}
\def\maxheight{\ifdim\Gin@nat@height>\textheight\textheight\else\Gin@nat@height\fi}
\makeatother
% Scale images if necessary, so that they will not overflow the page
% margins by default, and it is still possible to overwrite the defaults
% using explicit options in \includegraphics[width, height, ...]{}
\setkeys{Gin}{width=\maxwidth,height=\maxheight,keepaspectratio}
% Set default figure placement to htbp
\makeatletter
\def\fps@figure{htbp}
\makeatother
\setlength{\emergencystretch}{3em} % prevent overfull lines
\providecommand{\tightlist}{%
  \setlength{\itemsep}{0pt}\setlength{\parskip}{0pt}}
\setcounter{secnumdepth}{-\maxdimen} % remove section numbering
\usepackage{booktabs}
\usepackage{longtable}
\usepackage{array}
\usepackage{multirow}
\usepackage{wrapfig}
\usepackage{float}
\usepackage{colortbl}
\usepackage{pdflscape}
\usepackage{tabu}
\usepackage{threeparttable}
\usepackage{threeparttablex}
\usepackage[normalem]{ulem}
\usepackage{makecell}
\usepackage{xcolor}
\ifLuaTeX
  \usepackage{selnolig}  % disable illegal ligatures
\fi
\IfFileExists{bookmark.sty}{\usepackage{bookmark}}{\usepackage{hyperref}}
\IfFileExists{xurl.sty}{\usepackage{xurl}}{} % add URL line breaks if available
\urlstyle{same} % disable monospaced font for URLs
\hypersetup{
  pdftitle={Assignment 2 - Airbnb prediction models},
  pdfauthor={Anne Valder},
  hidelinks,
  pdfcreator={LaTeX via pandoc}}

\title{Assignment 2 - Airbnb prediction models}
\usepackage{etoolbox}
\makeatletter
\providecommand{\subtitle}[1]{% add subtitle to \maketitle
  \apptocmd{\@title}{\par {\large #1 \par}}{}{}
}
\makeatother
\subtitle{Course: ECBS6067 - Prediction with Machine Learning for
Economists}
\author{Anne Valder}
\date{2023-12-02}

\begin{document}
\maketitle

\textbf{The report below summarizes pricing strategies for small and
mid-size apartments in Copenhagen. Using a data-driven approach, I
determine that the average price for apartments hosting 2 to 6 guests is
300 DKK per night, as predicted by the best-performing model. The
driving factors influencing apartment prices are the number of beds and
the neighborhood.} The first part of the report delves into data
preparation and pre-processing. Subsequently, the three distinct price
prediction models, along with all modeling decisions, are explained.
This is followed by a comparison of the results with predicted apartment
prices in London. In addition, to further assess the predictive power of
the models out of sample, predictions are made not only on a holdout set
but also on an additional data set covering a different time frame,
essentially mimicking live data. Finally, the report provides a detailed
evaluation of the model performance of the best-performing machine
learning model, utilizing variance-important measures such as Shapely
values and partial dependence plots.\textbackslash{}

\textbf{Data Preparation:}The data for apartments in Copenhagen is taken
from \href{http://insideairbnb.com/get-the-data/}{Inside Airbnb}. The
two selected data sets include the review date 24 September, 2023 and
the review date 29 December, 2022. After joining the data sets, I
proceed with the \emph{sample design} as follows. First, I drop
variables that are of no use to the prediction exercise, e.g.~variables
containing urls, or specifics about the host etc. Next, I make sure that
all variables are cleaned (i.e.~remove characters or symbols) and of the
correct type and class for the prediction exercise (e.g.~conversion to
binary, numeric and factors). Moreover, I encode the `price' variable to
have the right format (no commas for thousands). Furthermore, I
transform the variable `amenity', which originally contains a listing of
all kinds of amenities a certain Airbnb has, into dummy variables. In
order to prevent to obtain over 2000 dummy variables I group the most
important amenities together and create approximately 15 dummies. To
ensure that the sample relates to the policy question at hand I filter
for apartments that can house only 2 to 6 people and that correspond to
``standard'' property types for representative Airbnb prices. To
circumvent that the analysis is distorted by errors I drop extreme
values by making sure that the minimum number of nights is at least
equal to one. At last, I create the variable `days since first review'
which allows me to analyse approximately for how long a apartment has
been used as a Airbnb i.e.~a proxy for age.

In terms of \emph{label engineering} I assure that minimum and maximum
values are reasonable and drop extreme values above 15.000 DKK
(i.e.~2000 Euro per night) which are likely to be errors after carefully
considering the characteristics of those high-priced Airbnbs. Moreover,
I check whether a log-transformation is feasible, since it might be
interesting to analyse relative changes rather than changes per monetary
unit. However, the two graphs in the appendix show that transforming the
target variable to logs does not alter the shape of the distribution
significantly. Therefore, I continue to use the target variable in
levels.

Next, I turn to the \emph{feature engineering:} choices. First, I
inspect \emph{missing values}. Here I drop variables if they contain to
many missing values and are not of importance for further analysis
(e.g.~calender updated and license). Other variables like `bathrooms' or
`bedrooms', I impute with the help of descriptive variables like
`bathroom text' or approximate with the variable `number of beds' and
`accommodates'. Some variables that contain only a few missing variables
get the missing values replaced with the median of the non-missing
values (e.g.~n\_days\_since, review).

In the following I consider th \emph{functional form} of some of the
predictors. However, these feature modifications are only relevant for
the OLS (LASSO) model. The other two (machine-learning) models are
non-parametric algorithms, which should be able to find interactions
between variables and non-linear behavior. After visual (see examples in
appendix) and some basic regression analysis I add squared, cubic and
log terms for the following variables: accommodates, beds,
number\_of\_reviews, n\_days\_since, and review\_scores\_rating.

Last, I consider the \emph{interactions of predictors}. Again this is
only relevant for the parametric OLS (LASSO) model.I assume interactions
with f\_property\_type, f\_room\_type and accommodates and visualize
those (see box plots in appendix). The interaction follows the believe,
the impact of room type on the response variable (e.g., price) varies
depending on the specific property type, an interaction term can capture
these differential effects. For example, the effect of upgrading to a
higher room type might have different implications for prices in
different property types (e.g., apartments vs.~houses). Moreover, I
include interactions between each f\_property\_type or f\_room\_type and
all the amenity dummies, similar as to the London example. The final
data set (including data for review date September 23 only) thus
consists of 99 variables with 11880 observations. Regarding the target
variable the mean price per night of an Airbnb in Copenhagen is 1184
DKK, the maximum is 15000 DKK and the minimum price is 75DKK.Further
summary statistics of the final data set can be found in the markdown
file in the chunk: Summary statistics.

\textbf{Models:} I chose to conduct the analysis using first an
\emph{LASSO model}, second a \emph{Random forest model} and last a
\emph{Boosting algorithm}. In the following I will go trough each model
argue for its choice and modelling decisions and then compare their
performance. To begin the prediction exercise I start by separating my
data set several times. First, I filter so that it only contain the data
from the first review date (29 December, 2022). Second, I separate the
data into a holdout set for evaluation and a working data set to
estimate the three different models. The working data set gets further
divided into a training and a test set for cross validation when
`calibrating' the right model parameters. This gets done automatically
via the `train' function in the caret package, when used with
cross-validation (method = ``cv''). The first model is the \emph{LASSO
model}. I chose it in order to not fully depend on domain knowledge and
to not be too concerned with variable selection but instead let the
model itself choose the most important predictors automatically.
Moreover, this way potential overfitting is avoided. I set a tuning grid
to do cross validation and choose the value for lambda (between 0.5 and
1), the parameter that sets the strength of the variable selection. The
larger the lambda the more aggressive is the selection and thus fewer
variables are left in the regression. The best performing LASSO model
has a lambda of\ldots{} From all possible predictors (including
interactions, polynomials etc.) the model shrinks the weakest
coefficients to zero to reduce variance and ends up picking 61 out of 74
predictors (see `lasso\_coeffs\_nz' in the LASSO chunk in the markdown
file). The second model is the \emph{Random forest}.

\textbf{Task 2: Model evaluation 2 with mimicked live data}

\begin{itemize}
\tightlist
\item
  Please pick two dates , one before and one after lockdowns in that
  city.
\item
  Carry out the exercise before the lockdown
\item
  Take your selected model and use it on the post lockdown date
\item
  Compare the predictive power of the model . Discuss briefly in the
  report
\end{itemize}

\textbf{Task 3: Model evaluation 3 Machine Learning and visualization:}

\begin{itemize}
\tightlist
\item
  Consider your ML model (RF or Boosting
\item
  Explain features ' contribution to predicted values on average , using
  Shapley values
\item
  You shall use a SHAP method
\item
  You may use graphs , but the emphasis is on the discussion part.
\item
  This is worth relatively few points , do it only if everything else is
  already done
\item
  extra add: graph on cv rmse?
\end{itemize}

\hypertarget{references-adjust}{%
\subsection{References (Adjust)}\label{references-adjust}}

ChatGPT, Filter Excluded Property Types, November 21, 2023. Retrieved
from:
\url{https://chat.openai.com/share/c7ed8e6b-ffa8-46f0-97e4-672bff890d60}

Békés,G., Kézdi,G.(2021).R, Python and Stata code for Data Analysis for
Business, Economics, and Policy, ch13-used-cars-reg, GitHub repository,
\url{https://github.com/gabors-data-analysis/da_case_studies/tree/master/ch14-used-cars-log}

Békés, G., \& Kézdi, G. (2021). Data analysis for business, economics,
and policy. Cambridge University Press, chapter 13.

\hypertarget{appendix}{%
\subsection{Appendix}\label{appendix}}

\begin{verbatim}
## `stat_bin()` using `bins = 30`. Pick better value with `binwidth`.
\end{verbatim}

\includegraphics[width=0.5\linewidth]{A2_files/figure-latex/figures-side-1}
\includegraphics[width=0.5\linewidth]{A2_files/figure-latex/figures-side-2}

\begin{verbatim}
## `geom_smooth()` using formula = 'y ~ x'
## `geom_smooth()` using formula = 'y ~ x'
\end{verbatim}

\includegraphics[width=0.5\linewidth]{A2_files/figure-latex/figures-side2-1}
\includegraphics[width=0.5\linewidth]{A2_files/figure-latex/figures-side2-2}

\includegraphics[width=0.5\linewidth]{A2_files/figure-latex/figures-side3-1}
\includegraphics[width=0.5\linewidth]{A2_files/figure-latex/figures-side3-2}

\end{document}
